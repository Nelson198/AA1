% Setup -------------------------------

\documentclass[a4paper]{report}
\usepackage[a4paper, total={6in, 10in}]{geometry}
\setcounter{secnumdepth}{3}
\setcounter{tocdepth}{3}

\PassOptionsToPackage{hyphens}{url}
\usepackage{hyperref}

% Encoding
%--------------------------------------
\usepackage[T1]{fontenc}
\usepackage[utf8]{inputenc}
%--------------------------------------

% Portuguese-specific commands
%--------------------------------------
\usepackage[portuguese]{babel}
%--------------------------------------

% Hyphenation rules
%--------------------------------------
\usepackage{hyphenat}
%--------------------------------------

% Capa do relatório

\title{
	Aprendizagem Automática I
	\\ \Large{\textbf{Projeto de Avaliação}}
	\\ -
	\\ Mestrado em Engenharia Informática
	\\ \large{Universidade do Minho}
	\\ Relatório
}
\author{
	\begin{tabular}{ll}
		\textbf{Grupo}
		\\\hline
		PG41081 & José Alberto Martins Boticas
		\\
		PG41091 & Nelson José Dias Teixeira
	\end{tabular}
}

\date{\today}

\begin{document}

\begin{titlepage}
    \maketitle
\end{titlepage}

% Resumo

\begin{abstract}
	Este projeto de avaliação relativo à unidade curricular de Aprendizagem Automática I consiste, globalmente, na aplicação de uma das técnicas abordadas durante as aulas sobre um 
	conjunto de dados. O conjunto de dados mencionado previamente é escolhido sem qualquer tipo de restrição por parte dos elementos do grupo por forma a despoletar o interesse dos 
	mesmos durante a análise estatística dos dados presentes. Como tal, durante a execução deste trabalho prático (cuja unidade curricular integra o perfil de Ciência de Dados), 
	surge uma motivação extra na interpretação dos resultados obtidos.
\end{abstract}

% Índice

\tableofcontents

% Introdução

\chapter{Introdução} \label{intro}
\large{
    \section{Apresentação da base de dados escolhida}
	A base de dados escolhida pelos dois elementos que constituem este grupo diz respeito a registos clínicos ou hospitalares de \textit{Cleveland} com informação relativa à presença ou 
	ausência de doenças cardíacas. Nesta pode-se encontrar exatamente 14 atributos, como por exemplo a idade e o sexo de uma dada pessoa bem como registos da sua pressão arterial. 
	
	\section{Contextualização}
	Apesar dos progressos consideráveis na luta contra as doenças cardiovasculares, estas continuam a ser a principal causa de morte na Europa. A alimentação inadequada é responsável por 
	cerca de metade das mortes e da incapacidade causada pelas doenças cardiovasculares.
	
	Em Portugal, cerca de 35 mil portugueses morrem anualmente por doenças cardiovasculares, que continuam a ser a principal causa de morte e representam um terço de toda a mortalidade da 
	população, embora muitas dessas mortes e desse sofrimento prolongado pudessem ser evitados por uma mudança simples nos hábitos alimentares.
	
	Com estes dados, os elementos que compõem este grupo ficaram sensibilizados acerca deste tema e mostraram-se interessados em investigar os fatores que de facto influenciam o aparecimento 
	de doenças cardíacas.

	\section{Definição das variáveis}
	No que diz respeito às variáveis associadas a esta base de dados foi relativamente acessível executar a sua análise. Contudo, o significado dos nomes atribuídos a cada uma das mesmas 
	não é de todo trivial. Desta forma, exibe-se de seguida de forma mais detalhada o que cada uma das variáveis representa:
    \begin{itemize}
	    \item \textbf{age}: idade de uma determinada pessoa em anos;
		\item \textbf{sex}: sexo de uma determinada pessoa:
		\begin{itemize}
			\item 1 : masculino;
			\item 0 : feminino.
		\end{itemize}
	    \item \textbf{cp}: tipo de dor no peito (4 tipos):
	    \begin{itemize}
			\item 1 : angina típica;
			\item 2 : angina atípica;
			\item 3 : dor não anginal;
			\item 4 : assintomática.
		\end{itemize}
	    \item \textbf{trestbps}: pressão arterial em repouso (em \textit{mm} \textit{Hg} de uma determinada pessoa na admissão no hospital);
	    \item \textbf{chol}: medição do colesterol de uma determinada pessoa em \textit{mg/dl};
		\item \textbf{fbs}: açúcar no sangue em jejum > 120 \textit{mg/dl} da pessoa em causa:
		\begin{itemize}
			\item 1 : verdadeiro;
			\item 0 : falso.
		\end{itemize}
	    \item \textbf{restecg}: resultados eletrocardiográficos em repouso de uma pessoa:
	    \begin{itemize}
			\item 0 : normal;
			\item 1 : com anormalidade da onda \textit{ST-T};
			\item 2 : com provável ou definida hipertrofia ventricular esquerda pelo critério de Estes.
		\end{itemize}
	    \item \textbf{thalach}: frequência cardíaca máxima alcançada;
		\item \textbf{exang}: angina induzida pelo exercício:
		\begin{itemize}
			\item 1 : sim;
			\item 0 : não.
		\end{itemize}
	    \item \textbf{oldpeak}: depressão \textit{ST} induzida por exercício em relação ao repouso;
	    \item \textbf{slope}: a inclinação do segmento ST do pico do exercício:
	    \begin{itemize}
			\item 1 : ascendente;
			\item 2 : plano (constante);
			\item 3 : descendente.
		\end{itemize}
	    \item \textbf{ca}: número de vasos principais (0-3);
		\item \textbf{thal}: distúrbio sanguíneo denominado por talassemia:
		\begin{itemize}
			\item 3 : normal;
			\item 6 : defeito fixo;
			\item 7 : defeito reversível.
		\end{itemize}
	    \item \textbf{target}: doença cardíaca:
	    \begin{itemize}
			\item 1 : sim;
			\item 0 : não.
		\end{itemize}
    \end{itemize}
	
	Ainda relativamente às incógnitas presentes na base de dados foi possível identificar tanto as variáveis quantitativas como as variáveis qualitativas ou categóricas. 
	Apresenta-se de seguida o tipo de cada uma das variáveis presentes:
	\begin{center}
		\begin{tabular}{ | l | l | }
		\hline
		\textbf{Nome da variável} & \textbf{Tipo de variável} \\ \hline
		age & Quantitativa discreta \\ \hline
		sex & Qualitativa ordinal \\ \hline
		cp & Qualitativa ordinal \\ \hline
		trestbps & Quantitativa discreta \\ \hline
		chol & Quantitativa discreta \\ \hline
		fbs & Qualitativa ordinal \\ \hline
		restecg & Qualitativa ordinal \\ \hline
		thalach & Quantitativa discreta \\ \hline
		exang & Qualitativa ordinal \\ \hline
		oldpeak & Quantitativa contínua \\ \hline
		slope & Qualitativa ordinal \\ \hline
		ca & Quantitativa discreta \\ \hline
		thal & Qualitativa ordinal \\ \hline
		target & Qualitativa ordinal \\ \hline
		\end{tabular}
	\end{center}
	
	\section{Objetivo de análise}
	Perante esta base de dados, os elementos que integram este grupo procuram, com a informação disponível, prever se uma determinada pessoa possui ou não uma doença cardíaca. Consequentemente,
	a variável de interesse para realizar este estudo corresponde à incógnita \textit{target}.
	Para além deste objetivo, vamos também perceber se existem outros fatores relativos ao coração que permitem prever certos eventos cardiovasculares. 
	
	Após traçarmos o objetivo principal na análise sobre a base de dados em causa, surgiram algumas questões da nossa parte. Apresentam-se de seguida as mesmas:
	\begin{enumerate}
	    \item Quantas pessoas possuem uma doença cardíaca?
	    \item Que fatores influenciam o aparecimento de uma doença cardíaca?
	    \item ...
	\end{enumerate}
	
	Consequentemente, por forma a responder a estas questões, é necessário especificar um modelo estatístico que se adequa a este contexto. Como tal, na próxima secção deste documento, 
	é apresentado o modelo requerido.
}

\chapter{Metodologia}
\large{
	Tal como o nome deste capítulo indica, nesta secção do documento será feita a exposição do modelo adotado pelo grupo para a análise correta dos dados presentes. Como tal, o modelo 
	escolhido foi ...
}

\chapter{Resultados}

\chapter{Conclusão}
\large{
	Após a apresentação da metodologia e dos resultados obtidos durante análise estatística desta base de dados, dá-se por concluído a realização deste trabalho prático. Com o modelo 
	implementado foi possível dar resposta às perguntas inicialmente traçadas, extraindo informação relevante para a compreensão e interpretação do contexto em causa. Consequentemente, 
	foi também possível alcançar os objectivos delineados na introdução deste documento. De notar também que apesar da dimensão reduzida desta base de dados foi possível especificar um
	modelo adequado para o efeito.
}

\chapter{Webgrafia}
	\begin{itemize}
		\item \textit{Website} indicado pela docente:
		\par \textit{\url{https://www.kaggle.com/datasets}}
        \item \textit{Website} com o resumo da base de dados escolhida:
		\par \textit{\url{https://www.kaggle.com/ronitf/heart-disease-uci}}
		\item \textit{Website} com a informação oficial da base de dados escolhida:
		\par \textit{\url{https://archive.ics.uci.edu/ml/datasets/Heart+Disease}}
    \end{itemize}
\end{document}