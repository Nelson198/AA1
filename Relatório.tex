% Setup -------------------------------

\documentclass[a4paper]{report}
\usepackage[a4paper, total={6in, 10in}]{geometry}
\setcounter{secnumdepth}{3}
\setcounter{tocdepth}{3}

\PassOptionsToPackage{hyphens}{url}
\usepackage{hyperref}

% Encoding
%--------------------------------------
\usepackage[T1]{fontenc}
\usepackage[utf8]{inputenc}
%--------------------------------------

% Portuguese-specific commands
%--------------------------------------
\usepackage[portuguese]{babel}
%--------------------------------------

% Hyphenation rules
%--------------------------------------
\usepackage{hyphenat}
%--------------------------------------

% Capa do relatório

\title{
	Aprendizagem Automática I
	\\ \Large{\textbf{Projeto de Avaliação}}
	\\ -
	\\ Mestrado em Engenharia Informática
	\\ \large{Universidade do Minho}
	\\ Relatório
}
\author{
	\begin{tabular}{ll}
		\textbf{Grupo}
		\\\hline
		PG41081 & José Alberto Martins Boticas
		\\
		PG41091 & Nelson José Dias Teixeira
	\end{tabular}
}

\date{\today}

\begin{document}

\begin{titlepage}
    \maketitle
\end{titlepage}

% Resumo

\begin{abstract}
	Este projeto de avaliação relativo à unidade curricular de Aprendizagem Automática I consiste, globalmente, na aplicação de uma das técnicas abordadas durante as aulas sobre um 
	conjunto de dados. O conjunto de dados mencionado previamente é escolhido sem qualquer tipo de restrição por parte dos elementos do grupo por forma a despoletar o interesse dos 
	mesmos durante a análise estatística dos dados presentes. Como tal, durante a execução deste trabalho prático (cuja unidade curricular integra o perfil de Ciência de Dados), 
	surge uma motivação extra na interpretação dos resultados obtidos.
\end{abstract}

% Índice

\tableofcontents

% Introdução

\chapter{Introdução} \label{intro}
\large{
    \section{Apresentação da base de dados escolhida}
	A base de dados escolhida pelos dois elementos que constituem este grupo diz respeito aos relatórios de incidentes criminosos reportados pelo departamento policial de Boston 
	(BPD - \textit{Boston Police Department}) desde 14 de Junho de 2015 até ao momento. Estes documentos registam os detalhes em torno de um determinado incidente que foi respondido 
	ela polícia de Boston.
	
	\section{Contextualização}
	Hoje em dia grande parte dos cidadãos que integram a sociedade mundial questionam a intervenção da polícia nas suas cidades ou países. Esta incerteza reside no facto de não só
	a população considerar que a força policial não é suficientemente adequada para um determinado tipo de incidente criminoso como também a abordagem adotada pela mesma ser demasiado 
	violenta. Consequentemente, surge a seguinte questão: \textit{"De que forma é que podemos melhorar a intervenção das autoridades?"}. Por forma a dar resposta a esta pergunta, vamos 
	estudar as intervenções policiais na cidade de Boston, nos Estados Unidos da América, com o intuito de perceber em que tipo de ocorrências estes mais intervêm, verificando se a sua 
	intervenção é ou não eficaz. Desta forma, é possível avaliar se de facto a polícia de Boston foi ao não correta no tratamento dos incidentes crimonosos reportados e, consequentemente, 
	afirmar se é até possível melhorar a sua intervenção.
	
	\section{Definição das variáveis}
	Quanto às incógnitas presentes na base de dados foi possível identificar tanto as variáveis quantitativas como as variáveis qualitativas ou categóricas. É exibido de 
	seguida as mesmas:
    \begin{itemize}
	    \item \textbf{INCIDENT\_NUMBER}: número do incidente;
	    \item \textbf{OFFENSE\_CODE}: código do crime/incidente;
	    \item \textbf{OFFENSE\_CODE\_GROUP}: nome do grupo/categoria associado(a) ao código do crime/incidente;
	    \item \textbf{OFFENSE\_DESCRIPTION}: descrição associada ao código do crime/incidente;
	    \item \textbf{DISTRICT}: distrito de Boston onde ocorreu o incidente;
	    \item \textbf{REPORTING\_AREA}: número que identifica a área onde foi reportado o crime;
	    \item \textbf{SHOOTING}: variável que indica se houve ou não um tiroteio num determinado crime;
	    \item \textbf{OCCURRED\_ON\_DATE}: data e hora do incidente;
	    \item \textbf{YEAR}: ano do incidente;
	    \item \textbf{MONTH}: mês do incidente;
	    \item \textbf{DAY\_OF\_WEEK}: dia da semana do incidente;
	    \item \textbf{HOUR}: hora do incidente;
	    \item \textbf{UCR\_PART}: \textit{Universal Crime Reporting Part number};
	    \item \textbf{STREET}: rua onde ocorreu o crime.
    \end{itemize}
	
	Apresenta-se de seguida o tipo de cada uma das variáveis presentes:
	\begin{center}
		\begin{tabular}{ | l | l | }
		\hline
		\textbf{Nome da variável} & \textbf{Tipo de variável} \\ \hline
		INCIDENT\_NUMBER & Qualitativa ordinal \\ \hline
		OFFENSE\_CODE & Quantitativa discreta \\ \hline
		OFFENSE\_CODE\_GROUP & Qualitativa nominal \\ \hline
		OFFENSE\_DESCRIPTION & Qualitativa nominal \\ \hline
		DISTRICT & Qualitativa nominal \\ \hline
		REPORTING\_AREA & Quantitativa discreta \\ \hline
		SHOOTING & Qualitativa ordinal \\ \hline
		OCCURRED\_ON\_DATE & Qualitativa ordinal \\ \hline
		YEAR & Quantitativa discreta \\ \hline
		MONTH & Quantitativa discreta \\ \hline
		DAY\_OF\_WEEK & Qualitativa nominal \\ \hline
		HOUR & Quantitativa discreta \\ \hline
		UCR\_PART & Qualitativa nominal \\ \hline
		STREET & Qualitativa nominal \\ \hline
		\end{tabular}
	\end{center}
	
	De salientar que foram removidas 3 variáveis que representavam as coordenadas do local do incidente criminoso pois não acrescentavam grande interesse na análise estatística. 
	
	\section{Objetivo de análise}
	Dado o enorme número de registos presentes na base de dados surgiram algumas perguntas ou curiosidades sobre as quais queremos tomar conhecimento. Apresentam-se de seguida as mesmas:
	\begin{enumerate}
	    \item Quais os tipos de crime são mais comuns?
	    \item Onde e quando é que os diferentes tipos de crimes têm maior probabilidade de ocorrer?
	    \item A frequência dos crimes cometidos muda ao longo do dia? E ao longo da semana? E durante o ano?
	    \item Qual é a distribuição de ocorrências de incidentes criminosos observados por hora?
	    \item O dia da semana influencia a quantidade de ocorrências de crimes?
	\end{enumerate}
	
	Consequentemente, por forma a responder a estas questões, é necessário especificar um modelo estatístico que se adequa a este contexto. Como tal, na próxima secção deste documento, 
	é apresentado o modelo requerido.
}

\chapter{Metodologia}
    \section{Modelo adoptado}

\chapter{Resultados}

\chapter{Conclusão}

\chapter{Webgrafia}
	\begin{itemize}
		\item \textit{Website} indicado pela docente:
		\par \textit{\url{https://www.kaggle.com/datasets}}
        \item \textit{Website} com o resumo da base de dados escolhida:
        \par \textit{\url{https://www.kaggle.com/AnalyzeBoston/crimes-in-boston}}
        \item Informação oficial acerca da base de dados:
		\par \textit{\url{https://data.boston.gov/dataset/crime-incident-reports-august-2015-to-date-source-new-system}}
		\item Informação mais detalhada acerca do campo \textit{UCR\_PART} presente na base de dados:
		\par \textit{\url{https://en.wikipedia.org/wiki/Uniform\_Crime\_Reports}}
    \end{itemize}
\end{document}