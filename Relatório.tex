% Setup -------------------------------

\documentclass[a4paper]{report}
\usepackage[a4paper, total={6in, 10in}]{geometry}
\setcounter{secnumdepth}{3}
\setcounter{tocdepth}{3}

\usepackage{xcolor}

% Encoding
%--------------------------------------
\usepackage[T1]{fontenc}
\usepackage[utf8]{inputenc}
%--------------------------------------

% Portuguese-specific commands
%--------------------------------------
\usepackage[portuguese]{babel}
%--------------------------------------

% Hyphenation rules
%--------------------------------------
\usepackage{hyphenat}
%--------------------------------------

% Capa do relatório

\title{
	Aprendizagem Automática I
	\\ \Large{\textbf{Projeto de Avaliação}}
	\\ -
	\\ Mestrado em Engenharia Informática
	\\ \large{Universidade do Minho}
	\\ Relatório
}
\author{
	\begin{tabular}{ll}
		\textbf{Grupo}
		\\\hline
		PG41081 & José Alberto Martins Boticas
		\\
		PG41091 & Nelson José Dias Teixeira
	\end{tabular}
}

\date{\today}

\begin{document}

\begin{titlepage}
    \maketitle
\end{titlepage}

% Resumo

\begin{abstract}
    Este projeto de avaliação relativo à unidade curricular de Aprendizagem Automática I consiste, globalmente, na aplicação de uma das técnicas abordadas durante as aulas sobre um conjunto de dados. O conjunto de dados mencionado previamente é escolhido sem qualquer tipo de restrição por parte dos elementos do grupo por forma a despoletar o interesse dos mesmos durante a análise estatística dos dados presentes. Como tal, durante a execução deste trabalho prático (cuja unidade curricular integra o perfil de Ciência de Dados), surge uma motivação extra na interpretação dos resultados obtidos.
\end{abstract}

% Índice

\tableofcontents

% Introdução

\chapter{Introdução} \label{intro}
\large{
    \section{Apresentação da base de dados escolhida}
	A base de dados escolhida pelos dois elementos que constituem este grupo diz respeito aos relatórios de incidentes criminosos reportados pelo departamento policial de Boston (BPD - \textit{Boston Police Department}) desde 14 de Junho de 2015 até ao momento. Estes documentos registam os detalhes em torno do incidente em questão e que foi respondido pelos polícias de Boston.
	
	\section{Contextualização}
	
	
	\section{Definição das variáveis}
	Quanto às incógnitas presentes na base de dados foi possível identificar tanto variáveis quantitativas como variáveis qualitativas ou categóricas. É exibido de seguida as mesmas:
    \begin{itemize}
	    \item \textit{\textbf{INCIDENT\_NUMBER}}: número do incidente;
	    \item \textit{\textbf{OFFENSE\_CODE}}: código do crime/incidente;
	    \item \textit{\textbf{OFFENSE\_CODE\_GROUP}}: nome do grupo/categoria associado(a) ao código do crime/incidente;
	    \item \textit{\textbf{OFFENSE\_DESCRIPTION}}: descrição associada ao código do crime/incidente;
	    \item \textit{\textbf{DISTRICT}}: distrito de Boston onde ocorreu o incidente;
	    \item \textit{\textbf{REPORTING\_AREA}}: número que identifica a área onde foi reportado o crime;
	    \item \textit{\textbf{SHOOTING}}: variável que indica se houve ou não um tiroteio num determinado crime;
	    \item \textit{\textbf{OCCURRED\_ON\_DATE}}: data e hora do incidente;
	    \item \textit{\textbf{YEAR}}: ano do incidente;
	    \item \textit{\textbf{MONTH}}: mês do incidente;
	    \item \textit{\textbf{DAY\_OF\_WEEK}}: dia da semana do incidente;
	    \item \textit{\textbf{HOUR}}: hora do incidente;
	    \item \textit{\textbf{UCR\_PART}}: (...) \textcolor{orange}{ACABAR !}
	    \item \textit{\textbf{STREET}}: rua onde ocorreu o crime.
    \end{itemize}
	
	De salientar que foram removidas 3 variáveis que representavam as coordenadas do local do incidente criminoso pois não acrescentavam grande interesse na análise estatística. 
	
	\section{Objetivo de análise}
	Dado o enorme número de registos presentes na base de dados surgiram algumas perguntas ou curiosidades sobre as quais queremos tomar conhecimento. Apresentam-se de seguida as mesmas:
	\begin{enumerate}
	    \item Quais os tipos de crime são mais comuns?
	    \item Onde é que os diferentes tipos de crimes têm maior probabilidade de ocorrer?
	    \item A frequência dos crimes cometidos muda ao longo do dia? E ao longo da semana? E durante o ano?
	\end{enumerate}
	
	Consequentemente, por forma a responder a estas questões, é necessário especificar um modelo estatístico que se adequa a este contexto. Como tal, na próxima secção deste documento, é apresentado o modelo requerido.
}

\chapter{Metodologia}
    \section{Modelo adoptado}

\chapter{Resultados}

\chapter{Conclusão}

\chapter{Webgrafia}
    \begin{itemize}
        \item \textit{Website} indicado pela docente:
        \par \textit{https://www.kaggle.com/AnalyzeBoston/crimes-in-boston}
        \item Informação oficial acerca da base de dados:
        \par \textit{https://data.boston.gov/dataset/crime-incident-reports-august-2015-to-date-source-new-system}
    \end{itemize}

\appendix
\chapter{Observações}


\end{document}